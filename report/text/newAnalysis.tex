\chapter{Parental Control - Why we need it}
Today children spend more and more time watching TV and playing video games. Research show that there have been an increase in children's usage of electronic media during the later years. \citep{sundhedsstyrelsen}\\
Research also show that this increase in media usage have consequences for the children. It has been linked to sleep deprivation\citep{bmcPublicHealth}, lack of physical exercise\citep{bmcPublicHealth} and ADHD\footnote{ADHD: Attention Deficit/Hyperactivity Disorder}\citep{mangler}. \\
\\
In this report we give a suggestion towards a solution to control this problem, by extending the concept of Parental Control over electronic devices.

\section{Concepts}
In this section key concepts is explained, in order to easier understand the report.\\
\\
\textbf{Smart Home} can refer to multiple forms of improvement on a home. Most commonly is: home automation, environment friendly improvements and power saving improvements. In this report we will only regard Smart Home in relation to home automation.\\
\\
\textbf{Parental Control} is a concept most commonly found in televisions, computers and internet-browsers. They all serve to limit the access to inappropriate content, either in the form of adult content, or advanced features that should not be touched.

\section{Increased media usage - It is a problem}
As stated in the intro to this chapter, the increase in media usage has been linked to sleep deprivation, lack of physical exercise and ADHD. Which in turn leads to higher risks of getting type 2 diabetes and concentration issues.\\
Worse is that there seems to be an increase in the media usage of each generation. From an article from the Danish Health Department, they point out that over a 5 year period children between the age of 10 and 15, have had an increase in media usage from 1.57 hours a day, to 2.47 hours a day, an increase of more than 50\%.\citep{sundhedsstyrelsen}\\
\\
If this tendency is not dealt with, the risk for children getting type 2 diabetes will also rise. Our suggestion to help lower the media usage is to give the parents the proper tools to limit their child’s media usage in a fair way, which will be discussed in the next section.


\section{Parental Control - How to extend it}
As previously mentioned parental control is commonly found in televisions, computers and internet-browsers. But none of these tools is meant to limit the usage of a device totally, they are able to do so, but it will result in parents having to enter passwords for their children every time they need to use the device.\\
The idea behind this project, is to extend the normal parental control, so that a child has a limited usage of media, without having the constant interaction from parents. \\
For the idea to be an improvement upon the existing parental control, it must be able to give the parents control over their children's media usage without forcing the parents to be around constantly. But simply restricting a child's media usage is not enough, for the idea to work. We believe it is also necessary to inspire the child to physical activity, maybe by helping out at the house?\\
\\
This lead us to an idea that consists of the following features:

\begin{enumerate}
	\item A way to set up permissions for which media and when they can be accessed
	\item A way to set up rules to make exceptions to these permissions
	\item A way to monitor/limit the usage of media used by a child
	\item A way to uniquely identify the child in the physical world
	\item A way to inspire children to physical activity
\end{enumerate}

After we had figured out what the key features of the system was, we started brainstorming how to implement them.\\
To satisfy feature 1. and 2. we quickly decided to make an administration web-site, this was mainly due to the semester demands of building anything web related. Among other solutions could have been a desktop- or phone application.\\
\\
For feature 3. we came up with a few different solutions. One solution was to write code for a bunch of medias, that then would be able to work directly with the administration web-site, but this would force us to write code for every media device from every fabricator that we intended the usage of our idea on.\\
We therefore decided that it would be better to create a device that could turn the power on and off from a media instead. We are aware that this solution might influence some media in a bad way, but most media today is setup to handle a sudden disappearance of power.\\
\\
For feature 4. we discussed if a mobile phone would be a good way to identify children, but this would encourage children to further use of electronic media, simply by giving them a mobile phone in the first place. We therefor came up with the idea to use RFID or NFC, to create small tags, which could be embedded into jewelry or key chains to identify the child.\\
\\
For feature 5. we did some serious consideration of how to motivate children in a way that would go along with the idea of the parental control system and we came up with the idea to implement chores into the system. In the belief that if a sort of point system was used to determine how much time a child would be able to use electronic media per week. Then a way to obtain extra points could be to do chores. Meaning if the child was attaining a habit of spending a lot of time using electronic media, the only way to attain more time would be to do chores which could then consist of moving the lawn or other physical exercises.\\
\fixme{hvad med frihedaktiviteter som kan dækkes ind under rules?}
\\
This idea then boils down to these 3 key aspects:

\begin{itemize}
	\item A device to toggle the power of an electronic media
	\item An individual key for each child that can be read by the device
	\item A web-interface to set up rules, permissions and view/modify allowed usage of electronic media
\end{itemize} 

Next follows a formal description of the problem, in the form of the problem statement followed by a more detailed explanation of how the system is designed and implemented.
