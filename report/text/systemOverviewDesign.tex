\chapter{System Overview Design}

%Tell how we intend the users to use it, both parents and children
%Symbolise use: TV, Controller, Reader, Tag, Child, Network Con, Laptop, Parent, Server
%
%Pull the list from the analysis and explain what we need to implement these features.
%\begin{enumerate}
	%\item A way to set up permissions for which media and when they can be accessed
	%\item A way to set up rules to make exceptions to these permissions
	%\item A way to monitor/limit the usage of media used by a child
	%\item A way to uniquely identify the child in the physical world
	%\item A way to inspire children to physical activity
%\end{enumerate}
%
%After we had figured out what the key features of the system was, we started brainstorming how to implement them.\\
%To satisfy feature 1. and 2. we quickly decided to make an administration web-site, this was mainly due to the semester demands of building anything web related. Among other solutions could have been a desktop- or phone application.\\
%\\
%For feature 3. we came up with a few different solutions. One solution was to write code for a bunch of medias, that then would be able to work directly with the administration web-site, but this would force us to write code for every media device from every fabricator that we intended the usage of our idea on.\\
%We therefore decided that it would be better to create a device that could turn the power on and off from a media instead. We are aware that this solution might influence some media in a bad way, but most media today is setup to handle a sudden disappearance of power.\\
%\\
%For feature 4. we discussed if a mobile phone would be a good way to identify children, but this would encourage children to further use of electronic media, simply by giving them a mobile phone in the first place. We therefor came up with the idea to use RFID or NFC, to create small tags, which could be embedded into jewelry or key chains to identify the child.\\
%\\
%For feature 5. we did some serious consideration of how to motivate children in a way that would go along with the idea of the parental control system and we came up with the idea to implement chores into the system. In the belief that if a sort of point system was used to determine how much time a child would be able to use electronic media per week. Then a way to obtain extra points could be to do chores. Meaning if the child was attaining a habit of spending a lot of time using electronic media, the only way to attain more time would be to do chores which could then consist of moving the lawn or other physical exercises.\\
%\fixme{hvad med frihedaktiviteter som kan dækkes ind under rules?}
%\\
%This idea then boils down to these 3 key aspects:
%
%\begin{itemize}
	%\item A device to toggle the power of an electronic media
	%\item An individual key for each child that can be read by the device
	%\item A web-interface to set up rules, permissions and view/modify allowed usage of electronic media
%\end{itemize} 
%Symbolise server: API, Website, Daemon, Server, connection to controller, connection to laptop

%Go into detail about rules, chores

%Explain difference of fields, explain each fields role without explaining how they are designed

%Tell how we decided each features placement

