\chapter{Overall System Design}
In this chapter we present the overall design of Media-Online Management(MOM). First we will present how we intend the users to use MOM. Then general concepts will be introduced which is an essential part of the system and will be mentioned several times hereafter. Thirdly the requirements of MOM will be explained in more detail. Finally the system architecture will be presented.    


\section{Media-Online Management} %name our system: temp name: Media-Online Management or MOM
The Media-Online Management (MOM) is the system that is our solution on the problem statement. To give an overview of the complete system a rich picture \citep{OOAD} have been made, see the figure \ref{fig:systemoverview}. The figure shows a home environment with a TV (media), computer and an internet connection, and it shows a server. 

There are two main use patterns. The first is a parent who manage their MOM using the website from the PC to add, change or delete system settings. This is pictured in the bottom left-side corner in figure \ref{fig:systemoverview}. 

The second use pattern is pictured in the upper left corner of the figure. It is a child user who wants to use a media which in this case is the television. But to watch television it need power and its power source is blocked by the controller. So the child needs to scan his tag and then the controller sends a message to the server which then reply whether the television can be turned on or not. When the child is done he must scan again such that points can be withdrawn from his user profile. If the child does not have any points he cannot turn on the television nor can he turn it on if a rule or permission does not allow it. If a parent wants to watch television without being restricted in any way he can make a rule that gives him unlimited access, but he would still need to scan his tag before and after using the television. \fixme{Vil vi lave en lille evaluering om vi synes det er fair at forældrene også skal scanne?}

\begin{figure}
	\centering
		\includegraphics[width=1.00\textwidth]{images/systemoverview2.png}
	\caption{system overview}
	\label{fig:systemoverview}
\end{figure}

On the server there will be a database, files that generate the website, a web service, and a daemon. These elements of the system will be explained further in section \vref{sec:RequirMOM} and after that the system architecture can be presented in section \vref{sec:sysArchitecture}. However, we need to explain some of the concepts that will be used in connection with this system.

\chapter{General Design Concepts}
\label{chapter:concepts}
%Go into detail about rules, permission, chores
In the Media-Online Management there are a few concepts that will be used through out the report. These are chore, permission and rules. The meaning of each will be explain in the following sections.

\section{Chore}
A chore in MOM is a representation of a house chore that is to be done regularly. We have included chore into this system to encourage children to help with the house chores which gives more time for media as a reward. Therefore each chore needs a number of points which will be given when the child has done the chore. 
An example on a house chore could be to take out the garbage, then the chore in MOM would have a name: ``take out garbage'', possible a bigger description of the chore: ``remember to sort the garbage into the correct trash cans'' and when the chore is done a number of points are given: ``10''.  

One disadvantage about the chore design is that the parent needs to use the website to award the child with points for doing a chore. It would have been better to automate it further, but to limit the scope of the project this would be future work.  
  
%is a subtype of rules. But we like to differentiate on rules not being necessary for the usage of the system, where permissions is. Permissions is only used to tell the system the normal allowed usage of media and rules is there to overwrite these permissions in special cases, or if the user wants to customize the system.\\



\section{Rule}
\label{sec:rule}
In the system a rule can be many things. The following are just a few examples of rules:

\begin{itemize}
	\item Susan has access to use the PC from 15:00 to 17:00 each day.
	\item The first Monday in a month Peter's points is increased by 100
	\item Mom and Dad's profiles have unlimited time and unrestricted access to any media
\end{itemize}

The rule is included because it gives the parents a more powerful method to control their children's use of medias. 
But there is one disadvantage with rules, it might be complicated for the user to understand how to use them. 
This means we must take care to design the concept of rules, so that it is easy to use and understand their effects.

A rule should be connected to one or more user profiles since the rules otherwise would be unnecessary. A rule consist of a name, a set of actions and  a restricted set of conditions depending on the actions. A condition is used to decide when a rule is relevant which is when all conditions fits. An action is something that can or should be done if all of the rule's conditions hold.
To see a quick overview of the different conditions and their name:

\begin{description}
	\item[Timeperiod] the action can be done in this time period.% \fixme{Does this still exist?} yes
	\item[Controller on] the action can be done if a specific controller is turned on. 
	\item[Controller off] the action can be done if a specific controller is turned off. 
	\item[True] The action can always be taken. This is actually kind of a time period that matches all of the time.
\end{description}

An overview of the actions and their name is listed below:

\begin{description}
	\item[Block user] it will block all of the profiles connected to the rule. 
	\item[Add points] it will add points to the profile.
	\item[Delete points]  it will delete points from the profile.  
	\item[Set maximum of point] it will set a maximum number of points that a profile can have. 
	\item[Unlimited time] it will give the profile unlimited time to be spend on any media. 
	\item[Access any controller] it will give the profile access to any media in the system. 
	\item[Cannot access any controller] the profile will not have access to any media. 
	\item[Access controller] it will give the profile access to a specific media. 
	\item[Cannot access controller] it will block the user from using a specific media. 
\end{description}
\fixme{Bette describtion of differens between Block user and Cannot access any controller}
Two designs have been made over rule. The first design is explained in the appendix \vref{appendixFirstRuleDesign}. The final design is of the rule is explained later in this section. The main difference is that in the first design is an extra type of condition timestamp. But late into the implementation of MOM we found that it would not be needed and therefor reworked the design, into what follows.\\
\\
The rule's structure is presented in a grammar in listing	\ref{grammar2} expressed in Extended Backus-Naur Form(EBNF) \citep{CoPL}.
A nonterminal is en-captured in $<>$ and a terminal is just a word or en-captured in ``''. 
Also the grouping is used represented in $()$, the replica symbol is $*$, comments is $(**)$ and alternative is $|$. 

A rule consist of a name and one of five action and condition sets which determine which actions and conditions can be combined. A rule can have several actions and conditions but only from the same set, see line 1-7. The action set is presented from 9-23 where all has a specific name, some have a specific Controller represented as a number and some has points which is a number. The condition set likewise represented from 9-23 and the condition types are presented from 25-30. There are three types but they each have a specific name. One type has a Controller, another only the name ``True'' and the last has a timeperiod. The timeperiod has two timestamps and if it is repeatable it has a string representation of the weekdays and a representation of how it is repeatable. 

\begin{lstlisting}[label=grammar2, caption=Grammar of a rule in EBNF]
<Rule>:= <name> (
	 	  (<ActionsetSet1><ConditionSet1>)*
		| (<ActionsetSet2><ConditionSet2>)*
		| (<ActionsetSet3><ConditionSet3>)*
		| (<ActionsetSet4><ConditionSet4>)*
		| (<ActionsetSet5><ConditionSet5>)*
	)

<ActionsetSet1> := "Block user" 
<ConditionSet1> := <ConditionTimeperiod> |<ConditionTrue>

<ActionsetSet2> := ("Add points" | "Delete points") <Points>
<ConditionSet2> := <ConditionTimeperiod>

<ActionsetSet3> := "Set maximum of point" <Points>
<ConditionSet3> := <ConditionTrue>

<ActionsetSet4> := "Unlimited time" | ("Access any controller" | "Cannot access any controller") <Controller>
<ConditionSet4> := <ConditionTimeperiod> | <ConditionTrue> 

<ActionsetSet5> := ("Access controller" | "Cannot access controller")<Controller>
<ConditionSet5> :=	<ConditionTimeperiod> |	<ConditionTrue>	|	<ConditionElse>	
				
<ConditionTimeperiod> :=  "TimePeriod" <Timestamp> <Timestamp> <Repeatable>					
<Repeatable> :=   <Weekdays> <Repeat> | Nill

<ConditionTrue> := "True" 
<ConditionElse>:= ("Device on" | "Device off") <Controller>
				
<Name> := ALPHA*  (* Upper and lowercase ASCII letters (A-Z,a-z) *)
<Controller> := DIGIT* (* Decimal digits (0-9) *)
<Points> := DIGIT*
<Timestamp> := 4*DIGIT,"-",2*DIGIT,"-",2*DIGIT, " ",2*DIGIT,":"2*DIGIT,":"2*DIGIT  (*YYYY-MM-DD HH:mm:ss*)
<Weekdays> := ("monday"| "tuesday"| "wednesday"| "thursday"| "friday"| "saturday"| "sunday")*
<Repeat> := "weekly" | "biweekly" | "triweekly" | "first in month" | "last in month"
\end{lstlisting}
		
\subsection{Examples of Rules}
In this section some examples of rules for a profile will be given.

The first example could be that the profile Peter gets point for each football training and the football training is Monday and Thursday from 18:00 to 19:30 each week. Then the action could be to add 25 points to Peter when condition holds. The condition is from 19:30 to 19:30 where the day is ``Monday, Thursday'' and it is repeated weekly. \\

The second example could be that Susan is grounded from the 2nd December to the 6th December both at 6 p.m.. In the system there will be a condition with a Timeperiod which is 2nd December 2013 18:00 to 6th December 18:00.
and an action that is ``Block user'' such that she cannot use any media or gain points in this period. \\

If the parents should make a rule which give themselves unlimited time and access to any media. The condition would be True and the actions would be ``Unlimited time'' and ``Access any controller''.\\


\section{Permission}
Permission in this system is whether a user have access to a given media. Permissions can also be expressed as a rule, but it is easier to make and understand a permission. This is also the main reason for differentiating the two for the user. 

A permission need a user and a controller. When a user then is connected to a permission it means this user has permission to use the media which is connected to this controller. An example could be that Peter has permission to use the TV. 

A permission is different from a rule in two ways according to the data structure. First, a permission does not need a condition, it is always true. Secondly a permission is always of the action type ``Access Controller''.\\
This means that permissions is only an abstraction making an easier understanding for the user.


\section{Permission and Rules Precedence}
It should be possible to override permissions and rules with another rule, but then some priority rules  need to be established. 
For the overriding of the two to be relevant they need to overlap in time which mean that the condition should use Timeperiod or True. Also if it is a rule then it should have an action with the name ``Cannot access controller'', ``Access controller'', ``Cannot access any controller'' or ``Access any controller'' to be relevant. So from the grammar \ref{grammar2} it is the rule sets $<name> <ActionsetSet4><ConditionSet4>$  and $<name> <ActionsetSet5><ConditionSet5>$ that need the priority.

It was decided that rules should always have precedence over permissions. But if the conflict is among two rules it is a more complex set of precedence rules. First to determine the precedence we look at the action's name. See figure \ref{fig:precendence} where the precedence graph is shown. Notice that the precedence for ``Cannot access controller'' and ``Access controller'' can be either way depending on the condition. Otherwise both of them has priority over ``Access any controller'' which also has priority over ``Cannot access any controller''. 
  
\begin{figure}
	\centering
		\includegraphics[width=0.75\textwidth]{images/precendence.jpg}
	\caption{Precedence of rules}
	\label{fig:precendence}
\end{figure}

When the condition determines the precedence we have different combinations:

\begin{itemize}
	\item If one is True and the other is Timeperiod then the rule with the Timeperiod has the higher precedence
	\item If both is True then Access controller has the precedence.
	\item If both is Timeperiod:
		\begin{itemize}
			\item If both is repeatable or non-repeatable%\fixme{First time we talk about non-repeatable} - not really look in grammar and text above, just saying if it is repeateble 
			then Access controller has the precedence
			\item If one is repeatable and the other is nonrepeatable then the rule with the nonrepeatable Timeperiod has the higher precedence.
		\end{itemize}
\end{itemize}

These precedence rules do not avoid all possible conflict but it limits them.  


This conclude the general concepts that will be used through out the remainder of the report. Next further details about the requirements of MOM is described.


\section{Requirements of MOM}
\label{sec:RequirMOM}

In the analysis we came to the conclusion that the following items is necessary for MOM: 
\begin{itemize}
	\item A device to toggle the power of an electronic media
	\item An individual key for each child that can be read by the device
	\item A website to set up rules, permissions and view/modify allowed usage of electronic media
\end{itemize} 


The first item is called controller in this system, the second is the tag and the last is the website on the computer. But also new components of the system is necessary. As explained in the previous section a daemon is needed to check the relevant rules and perform the appropriate actions if the conditions holds. Also during the following sections we find that a web service is needed. 

In the following sections the specification of the website, controller and tag, web service and daemon will be explained, before the design and implementation will be presented for each of them in the following chapters.  

\subsection{MOM's Website}
\label{subsection:momswebsite}
A Media-Online Management (MOM) owner needs a way to manage the system and for this a website will be made. There are several requirements for this website:

\begin{itemize}
	\item Register a new MOM together with a user profile which is a manager of this system
	\item Make it possible to add, edit and delete user profiles in an existing MOM 
	\item Make it possible to add, edit and delete Controllers from an existing MOM
	\item There should be a way to add, edit and delete Tags to the system. Also a tag should be connected to a user profile
	\item There need to be an option to add, edit and delete rules, permissions and chores from a system
	\item The user should be able to connect rules and permission with one or more user profiles. The connection should also be able to be removed without removing the rule or permission
	\item When a chore is done in the real world by a child profile then by connecting this profile to a chore, points should be added to his profile
\end{itemize}

There are also some requirements that is nice for the parents to use, but they are not essential for MOM system:

\begin{itemize}
	\item present the media consumption data as graphs or logs such that the parents easy can get an overview of their children's media consumption 
	\item present data from MOM in a calendar that shows rule and permission with profiles for whom this is relevant. The calendar should also show when a chore have been done and by who.
\end{itemize}

In chapter \fixme{insert ref} the design and implementation of the website will be presented in more detail. 
 

\subsection{Tag and Controller}
The tag used in this system is using Radio-frequency Identification (RFID). The tag need to be uniquely identified in MOM and it must uniquely identify its user. The tag is used in combination with the controller.

The controller is an Arduino which is connected to a tag reader. Like the tag it need to be uniquely identified in MOM. The controller must be able to do the following.

\begin{itemize}
	\item Read the data from the tag
	\item Send and receive messages from the server
	\item Control the power source of the media 
	\item Temporarily store the tag id that activated this controller
	\item Keep track of the time spent between activation of the media to its end
\end{itemize}
 
The design and implementation of the controller will be explained further in chapter \fixme{insert chapter ref}. 

The controller must communicate with the server and this is done via a web service.

\subsection{The web service}

The web service is used to parse the relevant data to the controller. Its requirements are:

\begin{itemize}
	\item Receive and send messages to the controller
	\item From a tag and controller it should be able to determine whether a user may use the media which is connected to the controller
	\item It must be able to subtract points from a user after he has been using a media
	\item It should able to calculate when a controller should be turned off because of a rule, permission or point.
\end{itemize}

The web service's design and implementation is explained further in chapter \fixme{ref to chapter}.


\subsection{Daemon}
The daemon is used to update data in the database from a rule which the user have made at some point. The daemon does not have any direct contact to a client. The daemon should handle any rule that is time determined so the condition is either Timestamp or Timeperiod. Then the daemon should do the appropriate action which could be add points to user or block a user.

The design and implementation of the daemon is presented in detail in chapter \fixme{ref to chapter}. 
 
\section{System Architecture}
\label{sec:sysArchitecture}
Now that all subsystem of the Media-Online Management have been presented the overall system architecture can be explained. MOM is using the client server architecture which is shown in the figure \ref{fig:serveroverview} \citep{OOAD}. This system has two types of client the first is a controller and the second is a computer where the user manages the system via the web site. The controller is depended on the web service on the server and the computer is depended on the web site. On the server there is the web site, web service and daemon which all are depended on the functions component, MySQLHelper class and data component. In the function component there are several functions to work on the data in the database and this component is depended on MySQLHelper because it should establish the database connection and parse the quires on to the database. 

\begin{figure}
	\centering
		\includegraphics[width=1.25\textwidth,  angle=90]{images/serveroverview.jpg}
	\caption{system architecture}
	\label{fig:serveroverview}
\end{figure}

In the following chapters design and implementation of the web site, web service, deamon and database will be presented.

 


