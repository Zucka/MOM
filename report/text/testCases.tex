\chapter{Test Suite}
\label{appen:testSuite}
This chapter contain all the testcases which has been tested in the Media-Online Management system.
%%Tests to Perform.

%testcases using the website and controller
These cases uses both the website and the controller:
%%Test 9.
\begin{table}[h]
	\centering
		\begin{tabular*}{\textwidth}{|l|l|}
		\hline
		\hline
		Name: & WS001\\
		\hline
		Description: & \parbox{0.70\textwidth}{Setup a complete system with a managing user, a regular user, a `TV Controller' and the accompanying rights to use it.}\\
		\hline
		Requirements: & \parbox{0.70\textwidth}{
		\begin{itemize}
			\item A computer with Internet access.
			\item The MOM website.
			\item Two Tags prepared with a Tag ID.
			\item An Arduino to function as the TV controller. 
		\end{itemize}}
		\\
		\hline
		Expected Results: & \parbox{.70\textwidth}{A managing user capable of logging into the TV Controller without loosing points. A regular User able to log into the TV controller while loosing points.}\\
		\hline
		Steps: & \parbox{.70\textwidth}{
		\begin{enumerate}
			\item Log into the MOM website with lniel10 and test.
			\item Attach the first Tag to the lniel10 profile.
			\item Add the permissions that enables the use of all devices without expending points.
			\item Create a profile 'Kevin' with the appropriate person information to act as a user.
			\item Attach the second tag to Kevin.
			\item Add controller TV into the system.
			\item Add the permissions to log into the TV controller.
			\item Perform Test AT001A on both profiles with addendum: Wait 3 minutes for both users and note if either expends points.
		\end{enumerate}}
		\\		
		\hline
		Result of Test: & \\
		\hline
		\end{tabular*}
\end{table}



%%Test 10.
\begin{table}[h]
	\centering
		\begin{tabular*}{\textwidth}{|l|l|}
		\hline
		\hline
		Name: & WS002\\
		\hline
		Description: & \parbox{0.70\textwidth}{Adding a rule to block a profile from using the 'TV' Device.}\\
		\hline
		Requirements: & \parbox{0.70\textwidth}{
		\begin{itemize}
			\item MOM Website.
			\item TV Device.
			\item Test Profile with Tag.
			\item A controller device for `TV'.
		\end{itemize}}
		\\
		\hline
		Expected Results: & \parbox{.70\textwidth}{The user attached to the profile will be unable to log into the `TV' device in accordance to the established Rule.}\\
		\hline
		Steps: & \parbox{.70\textwidth}{
		\begin{enumerate}
			\item Log into Mom Website with lniel10 and test.
			\item Add Rule to block the profile.
			\item Use the controller to test if you can activate the Device.
		\end{enumerate}}
		\\		
		\hline
		Result of Test: & \\
		\hline
		\end{tabular*}
\end{table}


%%Test 11.
\begin{table}[h]
	\centering
		\begin{tabular*}{\textwidth}{|l|l|}
		\hline
		\hline
		Name: & WS003\\
		\hline
		Description: & \parbox{0.70\textwidth}{Adding a rule to ensure that one device is turned on in order for another to be turned on.}\\
		\hline
		Requirements: & \parbox{0.70\textwidth}{
		\begin{itemize}
			\item A computer with Internet access.
			\item The MOM website.
			\item Web Browser with links to the API to simulate `TV'.
			\item A controller devices for `Playstation'.
			\item A Tags prepared with a Tag ID.
		\end{itemize}}
		\\
		\hline
		Expected Results: & \parbox{.70\textwidth}{The simulated devices one will have to be turned on in order to turn on the Simulated controller 2.}\\
		\hline
		Steps: & \parbox{.70\textwidth}{
		\begin{enumerate}
			\item If a `TV' device has not been established from earlier Test, create this.
			\item Create a `Playstation' device.
			\item Establish the Rule that the `Playstation' device cannot be turned on unless the `TV' device is.
			\item Use the browser to call turnOn for the `TV' and turn on the 'Playstation'. \footnote{The test controller have been made to the Playstation in the mean time}
			\item If the the `Playstation' did not turn on in step 4, turn on the `TV' and then try again to turn on the `Playstation'.
		\end{enumerate}}
		\\		
		\hline
		Result of Test: & \\
		\hline
		\end{tabular*}
\end{table}


%testcases to test only the controller
These cases only uses the controller:
%%Test 1
\begin{table}[h]
	\centering
		\begin{tabular*}{\textwidth}{|l|l|}
		\hline
		\hline
		Name: & AT001A\\
		\hline
		Description: & \parbox{0.70\textwidth}{Log in and out with the Arduino and a valid Tag.}\\
		\hline
		Requirements: & \parbox{0.70\textwidth}{
		\begin{itemize}
		  \item A LED light connected to the Arduino which functions as the ``Device''.
			\item Tag connected to a user with enough points and permission to use the Device.
			\item The Arduino running the final software version.
			\item Serial Connection to Arduino.
			\item Web Browser with link to Status API for the device being used.
		\end{itemize}}\\
		\hline
		Expected Results: & \parbox{.70\textwidth}{When the RFID antenna detects the tag, the LED light will turn on, The Serial will note that it is now running in State 1 and the web browser will report that the Status is green.
		When swiping the tag a second time, the LED light will turn off, The Serial will report that the Arduino is running at State 0 and the webbrowser will confirm that the status is RED for not running.
		After either swipe the Arduino will be ready for a new tag swipe.}\\
		\hline
		Steps: & \parbox{.70\textwidth}{
		\begin{enumerate}
			\item Turn on the Arduino. (Wait for Serial to confirm that the device is running.)
			\item Swipe tag over RFID antenna and observe if the LED turns on.
			\item On the Serial Output, note if the State changes from 0 to 1.
			\item Confirm on the web browser that the device is marked status:GREEN  running.
			\item Swipe tag over the RFID antenna again and observe if the LED turns off.
			\item On the Serial output, note if the State changes from 1 to 0.
			\item Confirm on the web browser.
		\end{enumerate}}
		\\

		\hline
		Result of Test: & \parbox{.70\textwidth}{First Iteration: Upon the first swipe the Arduino successfully logged in and changed state to 1 as expected. Confirmation also proved that to be status:GREEN. However, on the second swipe the Arduino would log out, but then immediately try to log in again with a corrupted Tag ID. All further attempts to log in would likewise fail do to Tag ID corruption.\\
		We traced the problem of the corrupted tag ID to the lack of a null character '\0' in the char array that held the ID's. However the Arduino would still try to log in again, which was resolved by implementing a strcmp(newID, oldID) to evaluate if the tag was different.\\
		Second Iteration: The Arduino performed as expected, succeding in logging users in and out.} \\
		\hline
		\end{tabular*}
\end{table}

%%Test 2.
\begin{table}[h]
	\centering
		\begin{tabular*}{\textwidth}{|l|l|}
		\hline
		\hline
		Name: & AT001B\\
		\hline
		Description: & \parbox{0.70\textwidth}{Log in with the Arduino and a Tag that does not have the proper permissions.}\\
		\hline
		Requirements: & \parbox{0.70\textwidth}{
		\begin{itemize}
		  \item A LED light connected to the Arduino which functions as the ``Device''.
			\item Tag connected to a user without the permission to use the Device.
			\item The Arduino running the final software version.
			\item Serial Connection To the Arduino.
			\item Web Browser with link to Status API for the device being used.
		\end{itemize}}
		\\
		\hline
		Expected Results: & \parbox{.70\textwidth}{When the RFID antenna detects the tag, the LED light will remain off, the state will remain 0 and the web browser will report that the Status is RED for not running.
		The Arduino will return to waiting for a new tag swipe.}\\
		\hline
		Steps: & \parbox{.70\textwidth}{
		\begin{enumerate}
			\item Turn on the Arduino. (Wait for Serial to confirm that the device is running.)
			\item Swipe tag over RFID antenna and observe if the LED turns on.
			\item On the Serial Output, note if the Arduino changes state.
			\item Confirm on the web browser that the device is marked status:RED for not running.
		\end{enumerate}}
		\\
		Result of Test: & The Arduino performed as expected and successfully declined logging in.\\
		\hline
		\end{tabular*}
\end{table}

%%Test 3.
\begin{table}[h]
	\centering
		\begin{tabular*}{\textwidth}{|l|l|}
		\hline
		\hline
		Name: & AT001C\\
		\hline
		Description: & \parbox{0.70\textwidth}{Log in with the Arduino and a Tag not supplied with enough points to run.}\\
		\hline
		Requirements: & \parbox{0.70\textwidth}{
		\begin{itemize}
		  \item A LED light connected to the Arduino which functions as the ``Device''.
			\item Tag connected to a user without enough points to use the Device.
			\item A serial connection to the Arduino
			\item The Arduino running the final software version.
			\item Web Browser with link to Status API for the device being used.
		\end{itemize}}
		\\
		\hline
		Expected Results: & \parbox{.70\textwidth}{When the RFID antenna detects the tag, the LED light will remain off, the state of the arduino will not change and the web browser will report that the Status is RED.		
		The Arduino will return to waiting for a new tag swipe.}\\
		\hline
		Steps: & \parbox{.70\textwidth}{
		\begin{enumerate}
			\item Turn on the Arduino. (Wait for Serial to confirm that the device is running.)
			\item Swipe tag over RFID antenna and observe if the LED turns on.
			\item Observe on the Serial Output if the state changes.
			\item Confirm on the web browser that the device is marked status:RED for not running.
		\end{enumerate}}
		\\
		\hline
		Result of Test: & The Arduino performed as expected successfully declined logging in.\\
		\hline
		\end{tabular*}
\end{table}

%%Test 4.
\begin{table}[h]
	\centering
		\begin{tabular*}{\textwidth}{|l|l|}
		\hline
		\hline
		Name: & AT001D\\
		\hline
		Description: & \parbox{0.70\textwidth}{Log in with the Arduino and a Tag not recognized.}\\
		\hline
		Requirements: &
		\parbox{0.70\textwidth}{
		\begin{itemize}
		  \item A LED light connected to the Arduino which functions as the ``Device''.
			\item A tag that has not been introduced to the system yet.
			\item A Serial Connection to Arduino.
			\item The Arduino running the final software version.
			\item Web Browser with link to Status API for the device being used.
		\end{itemize}}\\
				\hline
		Expected Results: & \parbox{.70\textwidth}{When the RFID antenna detects the tag, the LED light will remain off, the state of the arduino will not change and the web browser will report that the Status is RED.		
		The Arduino will return to waiting for a new tag swipe.}\\
		\hline
		Steps: & \parbox{.70\textwidth}{
		\begin{enumerate}
			\item Turn on the Arduino. (Wait for Serial to confirm that the device is running.)
			\item Swipe tag over RFID antenna and observe if the LED turns on.
			\item Observe on the Serial Output if the state changes.
			\item Confirm on the web browser that the device is marked status:RED for not running.
		\end{enumerate}}
		\\
		\hline
		Result of Test: & The Arduino performed as expected and successfully declined logging in.\\
		\hline
		\end{tabular*}
\end{table}

%%Test 5.
\begin{table}[h]
	\centering
		\begin{tabular*}{\textwidth}{|l|l|}
		\hline
		\hline
		Name: & AT002A\\
		\hline
		Description: & \parbox{0.70\textwidth}{Swipe a valid Tag that has the right permissions and points, while another user is logged in with the Arduino.}\\
		\hline
		Requirements: & \parbox{0.70\textwidth}{
		\begin{itemize}
		  \item A LED light connected to the Arduino which functions as the ``Device''.
			\item Tag connected to a user with enough points and permission to use the Device.
			\item A second Tag connected to a user with enough points and permission to use the Device.
			\item The Arduino running the final software version.
			\item Web Browser with link to Status API for the device being used.
		\end{itemize}}
		\\
		\hline
		Expected Results: & \parbox{.70\textwidth}{When the second tag is swiped while the first user is still active the LED should briefly flicker off and then on again as the new user logs back in.}\\
		\hline
		Steps: & \parbox{.70\textwidth}{
		\begin{enumerate}
			\item Turn on the Arduino. (Wait for Serial to confirm that the device is running.)
			\item Swipe tag over RFID antenna and observe if the LED turns on.
			\item Confirm on the web browser that the device is marked status:GREEEN for running.
			\item Swipe tag over RFID antenna and observe if the LED Turns off and then on again.
			\item Confirm on the web browser that the device is still marked status:GREEEN for running.
		\end{enumerate}}
		\\
		\hline
		Result of Test: & The Arduino performed as expected and successfully logged user 1 off before logging user 2 in due to the corrections performed after Test AT001A.\\
		\end{tabular*}
\end{table}
%%Test 6.
\begin{table}[h]
	\centering
		\begin{tabular*}{\textwidth}{|l|l|}
		\hline
		\hline
		Name: & AT003A\\
		\hline
		Description: & \parbox{0.70\textwidth}{Let Arduino run until getStatus is called without logged in User.}\\
		\hline
		Requirements: & \parbox{0.70\textwidth}{
		\begin{itemize}
			\item The Arduino running the final software version.
			\item Web Browser with link to Status API for the device being used.
			\item Serial Connection to Arduino.
		\end{itemize}}
		\\
		\hline
		Expected Results: & \parbox{.70\textwidth}{Run smoothly, remain in State 0, Remain turned off.}\\
		\hline
		Steps: & \parbox{.70\textwidth}{
		\begin{enumerate}
			\item Turn on the Arduino. (Wait for Serial to confirm that the device is running.)
			\item Wait and confirm that the status has run with the Serial Watch and note if its Status:RED.
			\item Confirm on the web browser that the device is still marked status:RED for running.
		\end{enumerate}}
		\\
		\hline
		Result of Test: & Initially we believed there was a bug where the Arduino would freeze after the getStatus being called the second time, but we have not since been able to reproduce it and after seeing the Arduino run unhindered for a hours at the time we conclude that it is running as expected.\\
		\hline
		\end{tabular*}
\end{table}
%%Test 7.
\begin{table}[h]
	\centering
		\begin{tabular*}{\textwidth}{|l|l|}
		\hline
		\hline
		Name: & AT003B\\
		\hline
		Description: & \parbox{0.70\textwidth}{Let Arduino run until getStatus is called with logged in User.}\\
		\hline
		Requirements: & \parbox{0.70\textwidth}{
		\begin{itemize}
			\item The Arduino running the final software version.
			\item Tag connected to a user with enough points and permission to use the Device.
			\item A serial connection to the Arduino.
			\item Web Browser with link to Status API for the device being used.
		\end{itemize}}\\
		\hline
		Expected Results: & \parbox{.70\textwidth}{The User will Remain logged in and the Arduino will not change from state 1.}
		\\
		\hline
		Steps: & \parbox{.70\textwidth}{
		\begin{enumerate}
			\item Turn on the Arduino. (Wait for Serial to confirm that the device is running.)
			\item Swipe tag over RFID antenna and observe if the LED turns on.
			\item Wait and confirm that the status has run with the Serial Watch and note if its Status:GREEN.
			\item Note if the Arduino changes state.
			\item Confirm that the LED remains on.
			\item Confirm on the web browser that the device is still marked status:GREEN for running.
		\end{enumerate}}
		\\
		\hline
		Result of Test: & The Arduino performed as expected and remained logged in after the getStatus had been called.\\
		\hline
		\end{tabular*}
\end{table}
%%Test 8.
\begin{table}[h]
	\centering
		\begin{tabular*}{\textwidth}{|l|l|}
		\hline
		\hline
		Name: & AT003C\\
		\hline
		Description: & \parbox{0.70\textwidth}{Let Arduino run until getStatus is called with logged in User, who has since logging in changed to not being permitted to use the Arduino.}\\
		\hline
		Requirements: & \parbox{0.70\textwidth}{
		\begin{itemize}
			\item The Arduino running the final software version.
			\item Tag connected to a user with enough points and permission to use the Device.
			\item Access to the \fixme{stuff that controls permissions.}
			\item A serial connection to the Arduino.
			\item Web Browser with link to Status API for the device being used.
		\end{itemize}}
		\\
		\hline
		Expected Results: & \parbox{.70\textwidth}{The Arduino will run with user logged in state 1 and with the LED turned on until the timer is reached. 
		Then the user will be logged out, the LED will turn off and the Arduino will move to state 0.}\\
		\hline
		Steps: & \parbox{.70\textwidth}{
		\begin{enumerate}
			\item Turn on the Arduino. (Wait for Serial to confirm that the device is running.)
			\item Swipe tag  over RFID antenna and observe if the LED turns on.
			\item Confirm on the web browser that the device is still marked status:GREEN for running.
			\item Use the Web Service to rescind permission to the device.
			\item Use browser to confirm that the device is marked as status:RED.
			\item Wait and confirm that the status has run with the Serial Watch and note if its Status:RED.
			\item Confirm that the LED turns off.
		\end{enumerate}}
		\\		
		\hline
		Result of Test: & \parbox{.70\textwidth}{First Iteration: It turned out that the back end API did not actually support this feature. This has since been rectified. \\Second Iteration: The Arduino now performs as expected.}\\
		\hline
		\end{tabular*}
\end{table}


To try everything else in the Website go though WS004:
%testing only website
%%Test 12.
\begin{table}[h]
	\centering
		\begin{tabular*}{\textwidth}{|l|l|}
		\hline
		\hline
		Name: & WS004\\
		\hline
		Description: & \parbox{0.70\textwidth}{Try each additional functionality of the web site}\\
		\hline
		Requirements: & \parbox{0.70\textwidth}{
		\begin{itemize}
			\item A computer with Internet access.
			\item The MOM website.
		\end{itemize}}
		\\
		\hline
		Steps: & \parbox{.70\textwidth}{
		\begin{enumerate}
			\item add chore 'vacume' with 10 point and add describtion 'the living room'
			\item add chore 'dishwashing' with 10 points.
			\item Kevin did chore 'dishwashing' give him points.
			\item delete 'vacume' and 'dishwashing'.
			\item one at the time edit something in tag, controller, and profile.
			\item one at the time and in this order delete ever rule, permission, tag, controller.
			\item delete 'Kevin'	
		\end{enumerate}}
		\\		
		\hline
		Result of Test: & \\
		\hline
		\end{tabular*}
\end{table}