\section{Time Checking}
\label{sec:timecheck}
Time checking in the in the Arduino Sketch is based on the native method called \verb|millis()|.
The \verb|millis()| method returns a long value with the number of milliseconds since the Arduino was turned on.
This number will continue to rise until the long overflows and the count starts once again from zero.\newline
The ability to check how much time has passed is needed for two vital functions.
\begin{itemize}
	\item Determining if there has passed five minutes since the last time \verb|getStatus()| was called, and by extension, if its time to call it again. This is done with the method \verb|checkTimeStatus()|.
	\item If the User has run out of available time to use, and the device needs to be turned off. This is done with the method \verb|checkNoTimeLeft()|.
\end{itemize}


\verb|checkTimeStatus()| can be seen in \autoref{CTS}, it uses the global variable |lastTime|, which is \verb|millis()| value from the last time that 5 minutes had passed and the \verb|getStatus()| method was called. In the off chance than \verb|millis()| has overflown and is a lower value than \verb|lastTime|, the code will establish the variable \verb|tempTime| to hold how many milliseconds was between \verb|lastTime| and the maximum unsigned Long value added together with the a amount of milliseconds passed since the overflow.

\begin{lstlisting}[frame=single, label=CTS, caption=Checking if five minutes has passed since the last getStatus() call.]
boolean checkTimeStatus()
{
  if(millis() < lastTime)
  {
    unsigned long tempTime = (0xffffffff - lastTime) + millis();
		//Check if tempTime Larger than Five Minutes.
    if(tempTime > 300000 )
    {      
      lastTime = millis();
      return true;
    }
    else
    {
      return false;
    }
  }
  else if ((millis() - lastTime) > 300000)
  {
    lastTime = millis();
    return true;
  }
  else
  {
    return false;  
  }  
}
\end{lstlisting}

The method \verb|checkNoTimeLeft()| in \autoref{CNTL} is very similar to the \verb|checkTimeStatus()| method. But rather than checking if five minutes have passed, it checks if there has passed more time than the user had left since the last time \verb|checkStatus()| was called, where the users \verb|timeleft| was last updated.

\begin{lstlisting}[frame=single, label=CNTL, caption=Checking if the user has spendt all his allotted time.]
boolean checkNoTimeLeft()
{
  
  unsigned long tempLeft = timeLeft * 60000;
  
  if(millis() < lastTime)
  {
    
    unsigned long tempTime = (0xffffffff - lastTime) + millis();
    
    if(tempTime > tempLeft )
    {      
      return true;
    }
    else
    {
      return false;
    }
  }
  else if ((millis() - lastTime) > tempLeft)
  {
    return true;
  }
  else
  {
    return false;  
  }  
}
\end{lstlisting}