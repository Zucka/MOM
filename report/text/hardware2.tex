\chapter{Hardware}
\label{chap:hardware}
In this section the hardware choices of this project will be talked about and analyzed. First the choice of hardware platform will be discussed, next the choice of hardware identification needed for the system and then the server setup. Followed each by a conclusion stating the chosen option.

\section{Hardware Platform}
\label{HardwarePlatform}
In order to control the power to a media(i.e. Television or Computer), some analog power control is needed. This can be controlled by an embedded microcontroller, the choice of hardware platform is, in our case, to pick the best microcontroller for the project. A development board is usually used during development . A development board is a single printed circuit board with a microcontroller and everything needed to get the microcontroller running. Every I/O\footnote{Input/Output}(generally General-purpose I/O) of the microcontroller is also made easily available using group pins. Picking a hardware platform is then condensed into picking a development board. Following is a comparison of a few different development boards.

\subsubsection{Arduino}
Arduino development boards come in a few variations, but common among all but one of them is that they are based on Atmel microcontrollers. Along with the development boards, Arduino also comes with its own software library \texttt{Wiring} which simplifies common I/O tasks. The Arduino development boards also implement a standard called \texttt{shields}, which are expansion circuit boards with pin headers that fit onto the development boards, with additional pins so that shields can be stacked. This makes the hardware platform easily extendable and flexible using this stacking system. Examples of shields could be GPS, Ethernet, LCD displays or breadboards.\citep{arduinoCC}


\subsubsection{Teensy}
Teensy, like Arduino is based on a Atmel microcontroller, but unlike Arduino is made to be as small as possible. Teensy also supplies an add-on library that makes Arduino programs compatible. Power is supplied via USB instead of a dedicated power input on the Arduino, which makes it cheaper. Teensy does not have an equivalent of shields.\citep{teensyPJRC}


\subsubsection{Raspberry Pi}
Raspberry Pi, unlike Arduino and Teensy, uses a ``full-sized'' ARM processor, which makes it more like a regular computer than a microcontroller. Despite that it has GPI/O pins just like Arduino and Teensy, which means that it has the capabilities needed for the project. Being ARM based means that the Raspberry Pi can also install operating systems such as Linux based operating systems. The ARM processor is also many times more powerful than the Atmel microcontrollers. This however comes with negatives such as higher power consumption, unused modules(GPU, sound or SD card storage) resulting in a larger board and a higher cost. \citep{raspberrypi}

\subsection{Conclusion - Platform}
In conclusion the platform chosen is Arduino. Arduino is chosen because of their shield design, which enables the use of a Ethernet shield as well as a RFID shield. Some of the group members has also already worked with Arduino, which means a lower learning curve.

\section{Hardware Identification}
In order for the person using the system to identify themselves, there has to be some form of hardware identification. This hardware identification has to be unique for that person, so that the system can calculate the allowed time the user can use on that media. Several factors have to be considered when choosing what hardware identification method to use, such as cost, user friendliness and security. Following is a comparison between some of the hardware identification methods available.

\subsubsection{RFID}
\label{rfidsect}
RFID, short for Radio-frequency Identification, is a method of using either passive or active tags for transferring data between a reader and a tag. A tag is an embedded chip with an antenna, that when receiving power transmits the data on the chip to any readers within range. Active tags have their own power source and thus can be scaled to emit their data over long distances. Passive tags on the other hand do not have an internal power source, and instead is powered by the reader via magnetic fields. This makes passive tags work on smaller ranges.\citep{rfidAndNfc} A tag can be compared to a bar code, with the difference being that tags scale much better when needing more data storage and the tag does not have to be visible, i.e. it can be embedded into for example key-cards, jewelry and such.

\subsubsection{NFC}
NFC, short for Near Field Communication, is a set of standards for the same technology that RFID uses. The differences between them is that NFC is a strict standard, uses only passive tags and only works over small distances.\citep{rfidAndNfc} NFC is also the technology being embedded in smartphones, which means that the projects system could be made compatible with apps.\\
\\
Fingerprints and iris recognition was also considered, but dismissed these solutions based on pricing and the fact that the human body changes a lot during the first 20 years of life.\citep{irisid}

\subsection{Conclusion - Identification}
In conclusion the identification chosen is RFID, because it could be provided by the university, however, originally NFC was preferred. 
The RFID has been chosen because it is easily accessible, and all of the practices used with RFID can be used with NFC. The reason for choosing NFC over RFID, is that it is a well defined standard, it is compatible with the technology used in smartphones and it is short range so no accidental reads happen. The reason for choosing NFC/RFID over fingerprint is that the security needed for fingerprints do not scale with simplicity and the possibility of a child's fingerprint changing too much. The reason for choosing NFC/RFID over iris is primarily the cost, as the cost of a high quality camera is too high since multiple cameras would be needed.

\section{Servers}
In order to maintain the wast amount of information collected from the users and in order to offer a graphical interface to add rules, permissions and chores to the system a server setup is needed.\\
Such server work can be done from a private home or centralized global server, maintained by the manufacturer of the system.\\

Listed here is the pros and cons of having a decentralized server setup versus centralized server setup.\\

\textbf{Decentralized Server Setup}\\
Pros:
\begin{itemize}
	\item Maintains proper functionality even if the internet connection out of the house is broken.
	\item Could be integrated in a Smart Home setup
\end{itemize}

Cons:
\begin{itemize}
	\item Each user will have to setup the system themselves or pay for a professional
  \item If the server malfunctions the user is the one with the problem
  \item Need to own a domain name and an external IP address in order to access the system from outside
\end{itemize}

\textbf{Centralized Server Setup}\\
Pros:
\begin{itemize}
	\item The setup of the system is for the user only registering on the website
	\item The user does not have to pay for extra electricity consumption of the server
	\item If the server fails, proper security measures can be setup to handle these problems
\end{itemize}

Cons:
\begin{itemize}
	\item If the internet connection malfunctions for what ever reason, the system will not work optimally.
  \item The user is limited to the functionality that is offered
\end{itemize}

As can be seen in the lists, most pros in the central setup is cons in the decentralized setup and vice verse.\\
This project focuses on developing a user friendly and upgradeable system, and therefore a centralized server setup has been chosen. The centralized system offers the easiest way for users to get started and it also makes it easier to deal with problems. The only problem that a user can face, is the lack of internet connection or power, which in our modern time is unlikely for a considerable amount of time.\\
Also the user would never have to do anything to keep their system up to date, since this would be handled on the server side.
