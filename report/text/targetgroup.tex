%	Who is the target group?
\chapter{new title to come}
In this chapter the target group is stated and it is explained why the target group would want to buy such a system.
 
\section{The Target Group}
This parental control system in a smart house is targeted at parents to children who are between 4 and 14.
We find the parents of children in the age group 0-3 to be less important since the children are too young to use the system.
Parent with children that are 15 or more are also less important because at that time the child might have bought his/hers own 
computer or television where the system should not be applied. Also the child would need the computer more for homework.  
    

\section{Children's Increased Media Usage}

There have been made many surveys that measure how much screen time children use on computers, television and console. These surveys 
have been made in several western countries, and their result typically indicates that children uses more 
time in front of the screen than either previous years or the recommended two hours. 
Even though this is general for many countries we only focus on the Danish marked and the Danish children.


%http://www.commonsensemedia.org/sites/default/files/research/zerotoeightfinal2011.pdf usa timer per day 3:46 children 5-8
%http://politiken.dk/tjek/forbrug/familieliv/ECE1779536/boern-bruger-syv-timer-daglig-foran-en-skaerm/ dk, 2,5 children 5-7
%http://www.biomedcentral.com/1471-2458/13/684 finland children 4th-6th grade (0:59 tv and 1:17 computer)

\subsection{Danish Children's media usage}
In 2012 a survey was made by TNS Gallup, where they looked into the 5-16 years old children's time usage of television, internet, computer games 
and console game. The average total time of the result can be found in table \ref{tab:Gallup2012screentime}. One point of critics of 
this survey is that it do not take into account that the children can use multiple media at the same time, so the actual number 
should be smaller. However, it can still be concluded from these results that even if most of the children uses two medias at a time
the exceed the recommended 2 hours per day. It can also be concluded that the time spend on the medias increases during the weekends and 
with age.

%cite: http://politiken.dk/tjek/forbrug/familieliv/ECE1779536/boern-bruger-syv-timer-daglig-foran-en-skaerm/

\begin{table}
\begin{center}
  \begin{tabular}{ | l | c | c | c | c || r | }
    \hline
				& 5-7 years & 8-10 years & 11-13 years & 14-16 years & In all\\ \hline
    Weekday 	&  		    &  			 &  		   &   			 & \\ 
	total time:  &	3:12	& 4:29		 &  6:13	   &  7:26		 & 5:23  \\ \hline
	Weekend &  &  &  &   & \\
	total time: &  	5:02	&  6:32		&  8:5		   &	9:15	& 7:17 \\ \hline
    \hline
  \end{tabular}
  \label{tab:Gallup2012screentime}
\end{center}
\end{table}

Seen from the general population and the states point of view this is a big problem because of the negative consequences that will follow 
too much media usage. 

\subsection{Consequences of Media Usage}
Two of the more common consequences of the children using the computer, console and television too much, is increasingly experience sleep disruptions 
and lack of physical exercise. Both of them also can lead to reducing in children's learning ability and other consequences which will be 
explained in the following sections.

 
\subsubsection*{Increasingly experiences sleep disruptions}
The child can experience disruption in his/hers sleep pattern. Typically the child that has media in his/hers room less that a child who do not. 
But also the usages of media have been proven to have an effect on the sleeping pattern. The sleep disruption can be more in form of too late bedtime
or disruption during sleep.
%cite http://www.sciencedaily.com/releases/2013/07/130725202325.htm 

The sleep disruptions can also lead to decreasing ability in visual and verbal memorizing which also have an effect on the
concentration difficulties in school. If the sleep disruption occurs too often then the child's health can be effected.
%cite http://pediatrics.aappublications.org/content/120/5/978.long



\subsubsection*{Lack of physical exercise}
The more the child is in front of a screen the less time is used on physical activities such as playing with other children, participating 
in sports or other recreational activities. Children in the age 5-17 years old are supposed to be physically active in an hour per day 
according to the Danish Board of Health. However, $2/3$ of the children do not meet this requirement and one of the reasons are 
the increased media usage.
%cite http://www.dr.dk/Nyheder/Indland/2013/09/29/235953.htm
This can also lead to serious health problems like obesity, diabetes and again concentration difficulties.  \\\\


Since too much media consumption is generally bad for the child it is important to understand why the parent do not control their media usage more, 
which is focused on in the next section.


\section{Parental Control - Ability and Tools} %senere overvej titlen

Generally weekdays of the parents are busy due to both parent's job, daily house work, and the children. However, the child has typically 
the school, maybe after-school care and recreational activities. Depending on the age of the child he/she maybe alone home in several hours 
before the parents get home. This is a time were the parents have no control of whether the child is using the television, console or computer. 
When the parents are at home the child might still be allowed to use the medias to much if the parents do or cannot keep track of their children's 
media usage.

The media rules and enforcement of them is probably different for each family, but something similar for the parents are the available tools to help 
them controlling the media usage. It should also be studied whether the parents feel they have the necessary tool to keep the children from
watching television and playing computer or console game.   

 
\subsection{our interviews}

more to come
