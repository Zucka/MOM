\chapter{Testing of the Media-Online Management}
There are different ways to evaluate the web interface e.g. an expert or user-based evaluation could be made. We would have preferred to make a user-based evaluation because this give a valuable input from a person who might use the system. However, to make such a sufficient user evaluation then we should find at least people with children which is between 5-14 years old and that would take too much time. A user-based evaluation of the entire MOM system would be optimal if the user and his/hers family could try it for a week or two before we interview them about the system. Also a log for our benefit could be made during this time to see how they use it of course with the test persons permission. Therefore to only evaluate the web interface we use an expert evaluation called Heuristic evaluation. 

\section{Heuristic evaluation in Theory}
In a Heuristic evaluation the test person(s) look for errors and potential usability errors which can be categorized in 12 item and divided into 3 groups\citep{DIEB}, which is shortly described below:
\begin{description}
\item[Learnability] make it ease for the user to learn and remembering the system:
\begin{itemize}
	\item Visibility - make it clear for the user which functions are available in the system and show what the system is doing.
	\item Consistency - be consistent in the design features and let it be similar to other comparable systems %similar systems/håber mening statid er der
	\item Familiarity - the language and symbols in the system should be familiar to the user.
	\item Affordance - let objects that look like something be it. E.g. if it looks like a button, it is a button. %pleas find en bedre sætning end 
	\fixme{'let objects that look like something be it' bedre forslag???}
\end{itemize}

\item[Effectiveness] is who ease it is for the user to use it and how safe it is to use:
\begin{itemize}
	\item Navigation - provide support to enable people to move around. Make clear navigation instructions.
	\item Control - make it clear who and what is in control, the user should be able to take control. There should be a connection between what happens in the system and what will happen outside of the system. 
	%is the mapping between control and effect. Relation what happens in system what happens in real life.
	\item Feedback - give feedback on what have happened after user do an action in the system.
	\item Recovery - how good it is to recover if the user makes an error. %e.g. user delete profile recover=> no
	\item Constraints - prevent people from doing inappropriate things in the system. This should be able to prevent serious error.
\end{itemize}

\item[Accommodation] accommodating differences between and respecting those differences:
\begin{itemize}
	\item Flexibility - let there be multiple ways of doing things.%=> dash bord if it worked
	\item Style - is the design attractive to the user. %who do it looks - attractive stylish???
	\item Conviviality - be polite and friendly. Do not use aggressive language.
\end{itemize}
\end{description}

Typically the test is done by 1-3 test persons\citep{HeuristicEvaluation}. The test person should be a usability experts or a software developer with the expertise in a relevant system. To make the test they should have relevant information of the user and a set of task to follow but they are also allowed to make their own.


\section{Heuristic Evaluation of Media-Online Management}
Before performing a Heuristic evaluation we need to determine what is needed before the evaluation is done and how the results should be presented.

\subsection{Planning the Evaluation}
When planing the evaluation the test persons need to get test persons. They need to get a small description of the user and some test cases. They should also get a list of the heuristics from which the system need to be evaluated from. This should all be made prior to the evaluation. \\

The evaluation of Media-online Management system is being done by (a) person(s)\fixme{når jeg ved mere (ental/flertal)} in the group which have been highly involved in the development of system. This is not optimal in this test where the test person(s) should be an usability expert from outside the development group.\\

The test persons also need to know the user. So the test person(s) need a general description of the user and their technology capabilities. The user of MOM is parents and children. The children is only in contact with the controller and the parents are both using the controller and the website. The parents has a general experience with the internet, but they are not experts. However we expect them to be able to at least set up an e-mail account with Google or Hotmail. The children has few technology skill but they are capable to turn on a normal television. \\

Before the test is conducted some test cases should be made the development team. This should cover several cases in the system such that the test person(s) is presented to every aspect of the system. The test cases of MOM can be found in section \vref{} \fixme{ref to be inserted}. \\


(- choose the Heuristics which it should be evaluated against) \fixme{to do}

\subsection{Presenting the Results}
\fixme{to do}

%testcases
\chapter{Test Suite}
\label{appen:testSuite}
This chapter contain all the testcases which has been tested in the Media-Online Management system.
%%Tests to Perform.

%testcases using the website and controller
The first cases are both using the website and the controller after that it is primarily the controller and the API which is being tested.
%%Test 9.
\begin{table}[h]
	\centering
		\begin{tabular*}{\textwidth}{|l|l|}
		\hline
		\hline
		Name: & WS001\\
		\hline
		Description: & \parbox{0.70\textwidth}{Setup a complete system with a managing user, a regular user, a `TV Controller' and the accompanying rights to use it.}\\
		\hline
		Requirements: & \parbox{0.70\textwidth}{
		\begin{itemize}
			\item A computer with Internet access.
			\item The MOM website.
			\item Two Tags prepared with a Tag ID.
			\item An Arduino to function as the TV controller. 
		\end{itemize}}
		\\
		\hline
		Expected Results: & \parbox{.70\textwidth}{Adding of a regular user,tags, a `TV Controller' and the accompanying rights to use it..}\\
		\hline
		Steps: & \parbox{.70\textwidth}{
		\begin{enumerate}
			\item Log into the MOM website with lniel10 and test.
			\item Attach the first Tag to the lniel10 profile.
			\item Add the permissions that enables the use of all media without expending points.
			\item Create a profile `Kevin' with 60 points and other appropriate person information to act as a user.
			\item Attach the second tag to Kevin.
			\item Add controller TV into the system.
			\item Add the permissions to log into the TV controller.
			\item Perform Test AT001A on both profiles with addendum: Wait 3 minutes for both users and note if either expends points.
		\end{enumerate}}
		\\		
		\hline
		Result of Test: & \parbox{.70\textwidth}{The adding of controller, tag and profile were done with ease and without error. However when trying to run it with the controller it coursed several problems due to the evaluation of rules which had been out date. This is a critical error which would prevent any real communication between the controller and API. So before continuing the test this need to be fixed. When the test was resumed it worked and the results from step 8 can be found in Test AT001A.} \\
		\hline
		\end{tabular*}
\end{table}



%%Test 10.
\begin{table}[h]
	\centering
		\begin{tabular*}{\textwidth}{|l|l|}
		\hline
		\hline
		Name: & WS002\\
		\hline
		Description: & \parbox{0.70\textwidth}{Adding a rule to block a profile from using the `TV' media.}\\
		\hline
		Requirements: & \parbox{0.70\textwidth}{
		\begin{itemize}
			\item MOM Website.
			\item TV Media.
			\item Test Profile with Tag.
			\item A controller for media `TV'.
		\end{itemize}}
		\\
		\hline
		Expected Results: & \parbox{.70\textwidth}{The user attached to the profile will be unable to log into the `TV' media in accordance to the established Rule.}\\
		\hline
		Steps: & \parbox{.70\textwidth}{
		\begin{enumerate}
			\item Log into Mom Website with `lniel10' and `test'.
			\item Add Rule to block the profile, try both with a timeperiod and true.
			\item Use the controller to test if you can activate the media.
		\end{enumerate}}
		\\		
		\hline
		Result of Test: &  \parbox{.70\textwidth}{This test failed at first when making the condition true, but it succeeded with a time period. The reason for the failure was due to some changes which happened late in the implementation in connection with the rules. It is a severe error because the user expect the user to be blocked, however the severity of the problem is lessened because of how likely users would use it or even find it. This lead to another observation as it is not clear how a rule can be made true which is a moderate error since it would be a waste of time for the user but they could find a way around it.}\\
		\hline
		\end{tabular*}
\end{table}


%%Test 11.
\begin{table}[h]
	\centering
		\begin{tabular*}{\textwidth}{|l|l|}
		\hline
		\hline
		Name: & WS003\\
		\hline
		Description: & \parbox{0.70\textwidth}{Adding a rule to ensure that one media is turned on in order for another to be turned on.}\\
		\hline
		Requirements: & \parbox{0.70\textwidth}{
		\begin{itemize}
			\item A computer with Internet access.
			\item The MOM website.
			\item Web Browser with links to the API to simulate `TV'.
			\item A controller for media `Playstation'.
			\item A Tag prepared with a Tag ID.
		\end{itemize}}
		\\
		\hline
		Expected Results: & \parbox{.70\textwidth}{The simulated controller 1 will have to be turned on in order to turn on the simulated controller 2.}\\
		\hline
		Steps: & \parbox{.70\textwidth}{
		\begin{enumerate}
			\item If a `TV' controller has not been established from earlier Test, create this.
			\item Create a `Playstation' controller.
			\item Establish the Rule that the `Playstation' controller cannot be turned on unless the `TV' controller is.
			\item Use the browser to call turnOn for the `TV' and turn on the `Playstation'. \footnote{The test controller have been made to the Playstation in the mean time}
			\item If the the `Playstation' did not turn on in step 4, turn on the `TV' and then try again to turn on the `Playstation'.
		\end{enumerate}}
		\\		
		\hline
		Result of Test: & \parbox{.70\textwidth}{This test was a complete failure since the rule could not be made and when trying the user gets the message `you are missing something'. This could cause much confusion for the user that would believe it was their mistake. There are no other way to make this kind of restriction so it is a severe error but the system could function without such a rule.}\\
		\hline
		\end{tabular*}
\end{table}


%testcases to test only the controller
%%Test 1
\begin{table}[h]
	\centering
		\begin{tabular*}{\textwidth}{|l|l|}
		\hline
		\hline
		Name: & AT001A\\
		\hline
		Description: & \parbox{0.70\textwidth}{Log in and out with the Arduino and a valid Tag.}\\
		\hline
		Requirements: & \parbox{0.70\textwidth}{
		\begin{itemize}
		  \item A LED light connected to the Arduino which functions as the ``media''.
			\item Tag connected to a user with enough points and permission to use the media.
			\item The Arduino running the final software version.
			\item Serial Connection to Arduino.
			\item Web Browser with link to Status API for the media being used.
		\end{itemize}}\\
		\hline
		Expected Results: & \parbox{.70\textwidth}{When the RFID antenna detects the tag, the LED light will turn on, The Serial will note that it is now running in State 1 and the web browser will report that the Status is green.
		When swiping the tag a second time, the LED light will turn off, The Serial will report that the Arduino is running at State 0 and the webbrowser will confirm that the status is RED for not running.
		After either swipe the Arduino will be ready for a new tag swipe.}\\
		\hline
		Steps: & \parbox{.70\textwidth}{
		\begin{enumerate}
			\item Turn on the Arduino. (Wait for Serial to confirm that the device is running.)
			\item Swipe tag over RFID antenna and observe if the LED turns on.
			\item On the Serial Output, note if the State changes from 0 to 1.
			\item Confirm on the web browser that the media is marked status:GREEN  running.
			\item Swipe tag over the RFID antenna again and observe if the LED turns off.
			\item On the Serial output, note if the State changes from 1 to 0.
			\item Confirm on the web browser.
		\end{enumerate}}
		\\
		\hline
		Result of Test: & see next page.\\
		\hline
		\end{tabular*}
\end{table}
		
\begin{table}[h]
	\centering
		\begin{tabular*}{\textwidth}{|l|l|}
		\hline
		\hline
		Name: & AT001A\\
		\hline
		Description: & \parbox{0.74\textwidth}{Log in and out with the Arduino and a valid Tag.}\\
		\hline
		Result of Test: & \parbox{.74\textwidth}{First Iteration: Upon the first swipe the Arduino successfully logged in and changed state to 1 as expected. Confirmation also proved that to be status:GREEN. However, on the second swipe the Arduino would log out, but then immediately try to log in again with a corrupted Tag ID. All further attempts to log in would likewise fail due to Tag ID corruption.
		We traced the problem of the corrupted tag ID to the lack of a null character `\textbackslash 0' in the char array that held the ID's. However the Arduino would still try to log in again, which was resolved by implementing a strcmp(newID, oldID) to evaluate if the tag was different. This error is a critical problem since it is expected of the controller that it should be able to receive multiple log-ins and log-outs during the day.\\
	Second Iteration: The Arduino performed as expected, succeeding in logging users in and out.} \\
		\hline
		\end{tabular*}
\end{table}

%%Test 2.
\begin{table}[h]
	\centering
		\begin{tabular*}{\textwidth}{|l|l|}
		\hline
		\hline
		Name: & AT001B\\
		\hline
		Description: & \parbox{0.70\textwidth}{Log in with the Arduino and a Tag that does not have the proper permissions.}\\
		\hline
		Requirements: & \parbox{0.70\textwidth}{
		\begin{itemize}
		  \item A LED light connected to the Arduino which functions as the ``media''.
			\item Tag connected to a user without the permission to use the media.
			\item The Arduino running the final software version.
			\item Serial Connection To the Arduino.
			\item Web Browser with link to Status API for the media being used.
		\end{itemize}}
		\\
		\hline
		Expected Results: & \parbox{.70\textwidth}{When the RFID antenna detects the tag, the LED light will remain off, the state will remain 0 and the web browser will report that the Status is RED for not running.
		The Arduino will return to waiting for a new tag swipe.}\\
		\hline
		Steps: & \parbox{.70\textwidth}{
		\begin{enumerate}
			\item Turn on the Arduino. (Wait for Serial to confirm that the device is running.)
			\item Swipe tag over RFID antenna and observe if the LED turns on.
			\item On the Serial Output, note if the Arduino changes state.
			\item Confirm on the web browser that the media is marked status:RED for not running.
		\end{enumerate}}
		\\
		\hline
		Result of Test: & \parbox{.70\textwidth}{The Arduino performed as expected and successfully declined logging in.}\\
		\hline
		\end{tabular*}
\end{table}

%%Test 3.
\begin{table}[h]
	\centering
		\begin{tabular*}{\textwidth}{|l|l|}
		\hline
		\hline
		Name: & AT001C\\
		\hline
		Description: & \parbox{0.70\textwidth}{Log in with the Arduino and a Tag not supplied with enough points to run.}\\
		\hline
		Requirements: & \parbox{0.70\textwidth}{
		\begin{itemize}
		  \item A LED light connected to the Arduino which functions as the ``media''.
			\item Tag connected to a user without enough points to use the media.
			\item A serial connection to the Arduino
			\item The Arduino running the final software version.
			\item Web Browser with link to Status API for the media being used.
		\end{itemize}}
		\\
		\hline
		Expected Results: & \parbox{.70\textwidth}{When the RFID antenna detects the tag, the LED light will remain off, the state of the arduino will not change and the web browser will report that the Status is RED.		
		The Arduino will return to waiting for a new tag swipe.}\\
		\hline
		Steps: & \parbox{.70\textwidth}{
		\begin{enumerate}
			\item Turn on the Arduino. (Wait for Serial to confirm that the device is running.)
			\item Swipe tag over RFID antenna and observe if the LED turns on.
			\item Observe on the Serial Output if the state changes.
			\item Confirm on the web browser that the media is marked status:RED for not running.
		\end{enumerate}}
		\\
		\hline
		Result of Test: & \parbox{.70\textwidth}{The Arduino performed as expected successfully declined logging in.}\\
		\hline
		\end{tabular*}
\end{table}

%%Test 4.
\begin{table}[h]
	\centering
		\begin{tabular*}{\textwidth}{|l|l|}
		\hline
		\hline
		Name: & AT001D\\
		\hline
		Description: & \parbox{0.70\textwidth}{Log in with the Arduino and a Tag not recognized.}\\
		\hline
		Requirements: &
		\parbox{0.70\textwidth}{
		\begin{itemize}
		  \item A LED light connected to the Arduino which functions as the ``Device''.
			\item A tag that has not been introduced to the system yet.
			\item A Serial Connection to Arduino.
			\item The Arduino running the final software version.
			\item Web Browser with link to Status API for the media being used.
		\end{itemize}}\\
				\hline
		Expected Results: & \parbox{.70\textwidth}{When the RFID antenna detects the tag, the LED light will remain off, the state of the arduino will not change and the web browser will report that the Status is RED.		
		The Arduino will return to waiting for a new tag swipe.}\\
		\hline
		Steps: & \parbox{.70\textwidth}{
		\begin{enumerate}
			\item Turn on the Arduino. (Wait for Serial to confirm that the device is running.)
			\item Swipe tag over RFID antenna and observe if the LED turns on.
			\item Observe on the Serial Output if the state changes.
			\item Confirm on the web browser that the media is marked status:RED for not running.
		\end{enumerate}}
		\\
		\hline
		Result of Test: & \parbox{.70\textwidth}{The Arduino performed as expected and successfully declined logging in.}\\
		\hline
		\end{tabular*}
\end{table}

%%Test 5.
\begin{table}[h]
	\centering
		\begin{tabular*}{\textwidth}{|l|l|}
		\hline
		\hline
		Name: & AT002A\\
		\hline
		Description: & \parbox{0.70\textwidth}{Swipe a valid Tag that has the right permissions and points, while another user is logged in with the Arduino.}\\
		\hline
		Requirements: & \parbox{0.70\textwidth}{
		\begin{itemize}
		  \item A LED light connected to the Arduino which functions as the ``media''.
			\item Tag connected to a user with enough points and permission to use the media.
			\item A second Tag connected to a user with enough points and permission to use the media.
			\item The Arduino running the final software version.
			\item Web Browser with link to Status API for the media being used.
		\end{itemize}}
		\\
		\hline
		Expected Results: & \parbox{.70\textwidth}{When the second tag is swiped while the first user is still active the LED should briefly flicker off and then on again as the new user logs back in.}\\
		\hline
		Steps: & \parbox{.70\textwidth}{
		\begin{enumerate}
			\item Turn on the Arduino. (Wait for Serial to confirm that the device is running.)
			\item Swipe tag over RFID antenna and observe if the LED turns on.
			\item Confirm on the web browser that the media is marked status:GREEEN for running.
			\item Swipe tag over RFID antenna and observe if the LED Turns off and then on again.
			\item Confirm on the web browser that the media is still marked status:GREEEN for running.
		\end{enumerate}}
		\\
		\hline
		Result of Test: & \parbox{.70\textwidth}{The Arduino performed as expected and successfully logged user 1 off before logging user 2 in due to the corrections performed after Test AT001A.}\\
		\hline
		\end{tabular*}
\end{table}
%%Test 6.
\begin{table}[h]
	\centering
		\begin{tabular*}{\textwidth}{|l|l|}
		\hline
		\hline
		Name: & AT003A\\
		\hline
		Description: & \parbox{0.70\textwidth}{Let Arduino run until getStatus is called without logged in User.}\\
		\hline
		Requirements: & \parbox{0.70\textwidth}{
		\begin{itemize}
			\item The Arduino running the final software version.
			\item Web Browser with link to Status API for the media being used.
			\item Serial Connection to Arduino.
		\end{itemize}}
		\\
		\hline
		Expected Results: & \parbox{.70\textwidth}{Run smoothly, remain in State 0, Remain turned off.}\\
		\hline
		Steps: & \parbox{.70\textwidth}{
		\begin{enumerate}
			\item Turn on the Arduino. (Wait for Serial to confirm that the device is running.)
			\item Wait and confirm that the status has run with the Serial Watch and note if its Status:RED.
			\item Confirm on the web browser that the media is still marked status:RED for running.
		\end{enumerate}}
		\\
		\hline
		Result of Test: & \parbox{.70\textwidth}{Initially we believed there was a bug where the Arduino would freeze after the getStatus being called the second time, but we have not since been able to reproduce it and after seeing the Arduino run unhindered for hours at the time we conclude that it is running as expected.}\\
		\hline
		\end{tabular*}
\end{table}

%%Test 7.
\begin{table}[h]
	\centering
		\begin{tabular*}{\textwidth}{|l|l|}
		\hline
		\hline
		Name: & AT003B\\
		\hline
		Description: & \parbox{0.70\textwidth}{Let Arduino run until getStatus is called with logged in User.}\\
		\hline
		Requirements: & \parbox{0.70\textwidth}{
		\begin{itemize}
			\item The Arduino running the final software version.
			\item Tag connected to a user with enough points and permission to use the media.
			\item A serial connection to the Arduino.
			\item Web Browser with link to Status API for the media being used.
		\end{itemize}}\\
		\hline
		Expected Results: & \parbox{.70\textwidth}{The User will Remain logged in and the Arduino will not change from state 1.}
		\\
		\hline
		Steps: & \parbox{.70\textwidth}{
		\begin{enumerate}
			\item Turn on the Arduino. (Wait for Serial to confirm that the device is running.)
			\item Swipe tag over RFID antenna and observe if the LED turns on.
			\item Wait and confirm that the status has run with the Serial Watch and note if its Status:GREEN.
			\item Note if the Arduino changes state.
			\item Confirm that the LED remains on.
			\item Confirm on the web browser that the media is still marked status:GREEN for running.
		\end{enumerate}}
		\\
		\hline
		Result of Test: & \parbox{.70\textwidth}{The Arduino performed as expected and remained logged in after the getStatus had been called.}\\
		\hline
		\end{tabular*}
\end{table}

%%Test 8.
\begin{table}[h]
	\centering
		\begin{tabular*}{\textwidth}{|l|l|}
		\hline
		\hline
		Name: & AT003C\\
		\hline
		Description: & \parbox{0.70\textwidth}{Let Arduino run until getStatus is called with logged in User, who has since logging in have his rights to use the media revoked.}\\
		\hline
		Requirements: & \parbox{0.70\textwidth}{
		\begin{itemize}
			\item The Arduino running the final software version.
			\item Tag connected to a user with enough points and permission to use the media.
			\item Access to the website MOM
			\item A serial connection to the Arduino.
			\item Web Browser with link to Status API for the media being used.
		\end{itemize}}
		\\
		\hline
		Expected Results: & \parbox{.70\textwidth}{The Arduino will run with user logged in, state 1, and with the LED turned on until the timer is reached. 
		Then the user will be logged out, the LED will turn off and the Arduino will move to state 0.}\\
		\hline
		Steps: & \parbox{.70\textwidth}{
		\begin{enumerate}
			\item Turn on the Arduino. (Wait for Serial to confirm that the device is running.)
			\item Swipe tag  over RFID antenna and observe if the LED turns on.
			\item Confirm on the web browser that the media is still marked status:GREEN for running.
			\item Use the Web Site to revoke permission to the media.
			\item Use browser to confirm that the media is marked as status:RED.
			\item Wait and confirm that the status has run with the Serial Watch and note if its Status:RED.
			\item Confirm that the LED turns off.
		\end{enumerate}}
		\\		
		\hline
		Result of Test: & \parbox{.70\textwidth}{First Iteration: It turned out that the API did not actually support any aspect of this feature. The error was if a permission granted the access then it was not checked whether he still had that permission. It is a moderate error since this can only happen once per permission lifetime. This has since been rectified. \\Second Iteration: The Arduino now performs as expected.}\\
		\hline
		\end{tabular*}
\end{table}



\section{Results of the Heuristic Evaluation}