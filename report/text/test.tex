\chapter{Testing of the Media-Online Management}
There are different ways to evaluate the web interface e.g. an expert or user-based evaluation could be made. We would have preferred to make a user-based evaluation because this give a valuable input from a person who might use the system. However, to make such a sufficient user evaluation then we should find at least people with children which is between 5-14 years old and that would take too much time. A user-based evaluation of the entire MOM system would be optimal if the user and his/hers family could try it for a week or two before we interview them about the system. Also a log for our benefit could be made during this time to see how they use it of course with the test persons permission. Therefore to only evaluate the web interface we use an expert evaluation called Heuristic evaluation. 

\section{Heuristic evaluation in Theory}
In a Heuristic evaluation the test person(s) look for errors and potential usability errors which can be categorized in 12 item and divided into 3 groups\citep{DIEB}, which is shortly described below:
\begin{description}
\item[Learnability] make it ease for the user to learn and remembering the system:
\begin{itemize}
	\item Visibility - make it clear for the user which functions are available in the system and show what the system is doing.
	\item Consistency - be consistent in the design features and let it be similar to other comparable systems %similar systems/håber mening statid er der
	\item Familiarity - the language and symbols in the system should be familiar to the user.
	\item Affordance - let objects that look like something be it. E.g. if it looks like a button, it is a button. %pleas find en bedre sætning end 
	\fixme{'let objects that look like something be it' bedre forslag???}
\end{itemize}

\item[Effectiveness] is who ease it is for the user to use it and how safe it is to use:
\begin{itemize}
	\item Navigation - provide support to enable people to move around. Make clear navigation instructions.
	\item Control - make it clear who and what is in control, the user should be able to take control. There should be a connection between what happens in the system and what will happen outside of the system. 
	%is the mapping between control and effect. Relation what happens in system what happens in real life.
	\item Feedback - give feedback on what have happened after user do an action in the system.
	\item Recovery - how good it is to recover if the user makes an error. %e.g. user delete profile recover=> no
	\item Constraints - prevent people from doing inappropriate things in the system. This should be able to prevent serious error.
\end{itemize}

\item[Accommodation] accommodating differences between and respecting those differences:
\begin{itemize}
	\item Flexibility - let there be multiple ways of doing things.%=> dash bord if it worked
	\item Style - is the design attractive to the user. %who do it looks - attractive stylish???
	\item Conviviality - be polite and friendly. Do not use aggressive language.
\end{itemize}
\end{description}

Typically the test is done by 1-3 test persons\citep{HeuristicEvaluation}. The test person should be a usability experts or a software developer with the expertise in a relevant system. To make the test they should have relevant information of the user and a set of task to follow but they are also allowed to make their own.


\section{Heuristic Evaluation of Media-Online Management}
Before performing a Heuristic evaluation we need to determine what is needed before the evaluation is done and how the results should be presented.

\subsection{Planning the Evaluation}
When planing the evaluation the test persons need to get test persons. They need to get a small description of the user and some test cases. They should also get a list of the heuristics from which the system need to be evaluated from. This should all be made prior to the evaluation. \\

The evaluation of Media-online Management system is being done by (a) person(s)\fixme{når jeg ved mere (ental/flertal)} in the group which have been highly involved in the development of system. This is not optimal in this test where the test person(s) should be an usability expert from outside the development group.\\

The test persons also need to know the user. So the test person(s) need a general description of the user and their technology capabilities. The user of MOM is parents and children. The children is only in contact with the controller and the parents are both using the controller and the website. The parents has a general experience with the internet, but they are not experts. However we expect them to be able to at least set up an e-mail account with Google or Hotmail. The children has few technology skill but they are capable to turn on a normal television. \\

Before the test is conducted some test cases should be made the development team. This should cover several cases in the system such that the test person(s) is presented to every aspect of the system. The test cases of MOM can be found in section \vref{} \fixme{ref to be inserted}. \\


(- choose the Heuristics which it should be evaluated against,(Visibility, Consistency, Control, Feedback, Familiarity is lisbeths suggestion) ) \fixme{TODO }


\subsection{Presenting the Results}
\fixme{to do}

%testcases
\section{Testcases}
In table \ref{tab:testcase} is one of the test cases that we made for testing the entire Media-Online Management. In the test case both apart of the Website and a controller are tested. The change been done in the web site should affect the controller.\\ 
 All test cases can be found in appendix \vref{appen:testSuite}.
%%Test 9.
\begin{table}[h]
\label{tab:testcase}
\caption{Example of a test cases}
	\centering
		\begin{tabular*}{\textwidth}{|l|l|}
		\hline
		\hline
		Name: & \parbox{0.6\textwidth}{WS001}\\
		\hline
		Description: & \parbox{0.70\textwidth}{Setup a complete system with a managing user, a regular user,  a `TV Controller' and the accompanying rights to use it.}\\
		\hline
		Requirements: & \parbox{0.70\textwidth}{
		\begin{itemize}
			\item A computer with Internet access.
			\item The MOM website.
			\item Two Tags prepared with a Tag ID.
			\item An Arduino to function as the TV controller. 
		\end{itemize}}
		\\
		\hline
		Expected Results: & \parbox{0.70\textwidth}{A managing user capable of logging into the TV Controller without loosing points. A regular User able to log into the TV controller while loosing points.}\\
		\hline
		Steps: & \parbox{0.70\textwidth}{
		\begin{enumerate}
			\item Log into the MOM website.
			\item Create a profile with appropriate personal information to act as a manager.
			\item Attach the first Tag to the new Manager profile.
			\item Add the permissions that enables the use of all devices without expending points.
			\item Create a profile with the appropriate person information to act as a user.
			\item Attach the second tag to the new user profile.
			\item Add the permissions to log into the TV controller.
			\item Perform Test AT001A on both profiles with addendum: Wait 3 minutes for both users and note if either expends points.
		\end{enumerate}}
		\\		
		\hline
		Result of Test: & \\
		\hline
		\end{tabular*}
\end{table}


\section{Collected Results of the Heuristic Evaluation}