\chapter{Problem Statement}
As the prevues chapter explained, children's increase in media usage is becoming a problem. This is a problem we want to solve by the usage of modern technology, internet applications and physical recognition of users.\\
In this chapter a formal definition of the problem is written together with an emphasis on what challenges this problem rises.\\


\begin{verse}
\textit{Parents are not able of helping their children administer their IT/TV consumption.\\
This results in their children getting a lessened learning ability, a bad sleeping pattern and a higher risk of type 2 diabetes.\\
How can hardware identification and webservices give parents the tools to help their children manage their media consumption?}
	\begin{itemize}
		\item How do we identify unique users in a subtle and child friendly way.
		\item How do we enforce restrictions on media devices.
		\item How do we facilitate concepts as rules, permissions and chores without parents interaction.
	\end{itemize}
\end{verse}


With this formal statement of the problem in mind, the next chapter goes into detail of how we intend to solve these challenges.

%English: Parents are not able of helping their children administer their IT/TV consumption.
%This results in their kids getting a lessened learning ability, a bad sleeping pattern and more.
%How can smart home technology give parents the tools they need to help their children administer their IT/TV consumption?

%Edited version:
%How can hardware identification and webservices give parents the tools to help their children manage their media consumption. 
%(Childrens media consumptions is increasing, giving higher risks of type 2 diabetes.