\subsection{Using RFID With The Arduino}
\label{sec:rfidsect}
\fixme{Mangler lidt mere start som der var i Using the ethernet shield.}
The Arduino board utilizes the SM130 by sending it commands and receiving replies through Digital Pins 7 and 8.
These pins are initialized using the library \verb|SoftwareSerial.h|.\\
\begin{lstlisting}
SoftwareSerial rfid(7, 8); //Sets up the Digital Pins 7 and 8 '
                           //Which the RFID reader Communicates through.
													
rfid.begin(19200); //This Method is called in the Setup() phase 
                   //of the Arduino Startup.
									 //It starts the Communication on the standard baud
									 //rate of 19200 bps, N, 8, 1.
											
\end{lstlisting}
\fixme{enten skal vi forklare hvad "`baud"' og hvad 19200 bps, N, 8, 1 står for.}		

These messages are a series of bytes sent one at the time in the form of Hexadecimals and in the context of an UART framed message. \newline
\begin{tabular}{|l|l|l|}
\hline \hline
Header: & 1 Byte & Must always be 0xFF.\\ \hline
Reserved: & 1 Byte & Must always be 0x00.\\ \hline
Message Length: & 1 Byte & The amount of bytes used on Command and Data.\\ \hline
Command: & 1 Byte & The Hexadecimal value of the command in question.\\ \hline
Data: & n Bytes & The bytes containing the data needed for the command.\\ \hline
Checksum: & 1 Byte & The combined value of all above hexadecimals.\\ \hline
\end{tabular}
The command messages needed we need to allow logging in with a tag are: 
\begin{itemize}
	\item Seek: The Module activates `anti-collision' and `Select Tag' so that it can catch a tag when it comes in range.
	\begin{itemize}
		\item Message Length:Seek has a message length of 1 byte as there is no accompanying data.
		\item Command: 0x82 is the Seek Command.
	\end{itemize}
	\item Authenticate: When a tag has been found this command authenticates a single block of data on the RFID Tag, which must be done before it is read.
	\begin{itemize}
		\item Message Length: The authenticate command message is 3 bytes long as it needs 2 databytes.
		\item Command: 0x85 is the Authenticate Command.
		\item Data Byte 1 (Block Number): Tells the SM130 which data block on the tag is to be authenticated.
		\item Data Byte 2 (Key Type): Informs which type of key should be used to authenticate with. The SM130 has 2 Key types to authenticate with (A and B), and 15 slots for each to store individual keys. We use the build in 0xFF which authenticates with Type A and the transport key ``FF FF FF FF FF FF''.
	\end{itemize}
	\item Read Block: Reads a previously Authenticated block of data on the RFID Tag, The data contains the Tag ID.
	\begin{itemize}
		\item Message Length: The Read Block Command message is 2 bytes long as it needs information on what block is to be read.
		\item Command: 0x86 is the Read Block Command.
		\item Data Byte (Block Number): Tells the SM130 which data block on the tag is to be read.
	\end{itemize}
\end{itemize}

As an example \autoref{RBCE} is the code run when sending the Read Block Command to the SM130.
\begin{lstlisting} [label=RBCE, caption=Example of the Read Block Command from the code point of view.]
void read_block_RFID(void)
{
  //Read Block Command in UART, sent to the SM130. The Block needs to be Authenticated beforehand.
  rfid.write((uint8_t)0xFF); //Header: 1 byte, must always be 0xFF.
  rfid.write((uint8_t)0x00); //Reserved: 1 byte, must always be 0x00.  
  rfid.write((uint8_t)0x02); //Message Length: 1 byte, both for Command and Data (Here: 2 bytes).
  rfid.write((uint8_t)0x86); //Command: 1 byte, 0x86 is the Read Block Command
  rfid.write((uint8_t)0x01); //Data(Block Number): 1 byte, read block no 0x01. 
  rfid.write((uint8_t)0x89); //The Message Checksum.
  delay(10); 
}													
\end{lstlisting}


On being given a command the SM130 will send a response from which you can tell if the command has been successfully performed or not.
In the code this information is saved on a char array by the parse method seen in \autoref{arduinoparse}. 
On `Seek' and `Read Block' we determine the success by reading the slot containing the length of the Message, while on the Authenticate command the Data slot is read to determine success.

\begin{lstlisting} [label=arduinoparse, caption=The method used to read the SM130's response.]
void read_block_RFID(void)
{
void parse_response(char PH[], int length)
{
  for(int i=1;i<length;i++)
  {
    PH[i] = 0;
  }
  
  while(rfid.available()) //This whileloop runs so long as there is still bytes to be read. 
  {
    if(rfid.read() == 255) //Checks for the Message Header.
    {
      for(int i=1;i<length;i++)
      {
        PH[i]= rfid.read(); //For the Length of the expected UART message, Add bytes to the Array.
      }
    }
  }
}
}													
\end{lstlisting}


\fixme{the table with byte code is moved to appendix ref is append:bytecode}