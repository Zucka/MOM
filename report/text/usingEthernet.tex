\subsection{Using the Ethernet Shield.}
\label{sec:ethernetshield}
%\fixme{Mangler kort kommentar om hvorfor Ethernat er brugt istedet for en Wifi connection} Er håndteret tidligere, i design afsnittet.
The Ethernet shield is the primary component in accessing the API as it allows for the Arduino to access the internet as a client.
The Ethernet shield is used along with a JSON parser in order to make HTML requests on the web site that contains the system API and decode its response.

The Ethernet Shield comes with a MAC address printed on the side which is used to automatically obtain an IP Address.
This along with the target server is established as global variables as seen in \autoref{EthStart} along with the Method call in \verb|Setup()| that establishes a connection with the LAN. All supplied from the Arduino Library \verb|Ethernet.h|.
%\fixme{kunne være korte statement på sær funktioner for eksemple F står for når du printer} Fjernet print linjen.
%\fixme{linje 22, Hvad sker der så, er det bare en død Audrino?} Forældet kommentar
\begin{lstlisting}[frame=single,language=C, label=EthStart, caption=The Method Calls used to establish and maintain a connection.]
//The Mac Address declared.
byte mac[] = { 0x90, 0xA2, 0xDA, 0x0E, 0xC5, 0x94 };

//Our target Server that the API is called on
char server[] = "spcadmin.tk";

//When this statement is evaluated the Arduino connects to the LAN.
//This happens during Setup();
if (Ethernet.begin(mac) == 0) 
{
  //If the Evaluation returns 0 it fails and stops doing anything.
	while(true); 
}

//The following code us run in every loop through checkConnection() call to assure
//that the DHCP lease is renewed when needed.
Ethernet.maintain();
\end{lstlisting}
\newpage
Whether the user tries to log in or out from the device, the Arduino will have to access the web server via the API.
The process is almost identical in either case and can be divided into two parts:
\begin{itemize}
	\item Connecting to the API.
	\item Parsing the servers Reply.
\end{itemize}

As seen in \autoref{pageConnect} the Device ID and the Tag ID is connected in a char array to form the url which complies with the action that is being performed. For more information on the API see chapter \vref{chap:api}.
Then in return the Ethernet Shield receives the HTML Page, and in \autoref{pageDecipher} the Arduino reads this information one byte at the time where all but the JSON string is stripped and stored in a char pointer.
When all the bytes have been through this process the connection is closed and the JSON string is parsed the result of the action. This can be seen in \autoref{pageJSON}.
\begin{lstlisting}[frame=single, language=C, label=pageConnect, caption=Connecting to the Server and creating an HTML request.]
void turnOn(void)
{
  if(client.connect(server, 80))
  {
    //Connected

    client.print(F("GET /api/api.php/turnOn/"));
    client.print(devID);
    client.print(F("/"));
    client.print(useID);
    client.println(F(" HTTP/1.1"));
    client.println(F("Host: spcadmin.tk"));
    client.println(F("Connection: close"));
    client.println();
\end{lstlisting}

\begin{lstlisting}[frame=single, language=C, label=pageDecipher, caption=Removing all but the important information from the website.]
while (client.connected())
{
  //Builds the JSON string from the data passed by the website.
  while(client.available()) 
  { 
    buff = client.read();   //Bytes are passed through the Ethernet Shield with client.Read();
    if(buff == '{')        //The JSON string starts with '{' and stops with '}'.
    {
      toggle = !toggle;
      *(on+i) = buff;
      i++;
    }
    else if(buff == '}')
    {
      toggle = !toggle;
      *(on+i) = buff;
      i++;
    }
    else if(toggle)
    {
      *(on+i) = buff;
      i++;
    }
  }
}
client.stop();
\end{lstlisting}

\begin{lstlisting}[frame=single, language=C, label=pageJSON, caption=The JSON Code Getting a value with a Token.]
aJsonObject* json = aJson.parse(on);
aJsonObject* statJson = aJson.getObjectItem(json,"status");
char * stat = statJson->valuestring;
\end{lstlisting}