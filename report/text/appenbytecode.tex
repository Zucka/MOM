\chapter{RFID Output}
\label{append:bytecode}
\begin{tabular}{|l|l|}
\hline
& The RFID output in UART for seeking for tag:\\
&  On `Tag Found'. (The length would be 0x06.)\\
\hline
Slot 0-3 & Contains the message ``Header'', ``Reserved'', ``Length'' and\\
& ``Command''.\\
Slot 4 & Contains the tag type. \\
Slot 5-8 & Contains the data stored in the block.\\
Slot 9 & Contains the Checksum.\\
\hline
&On `no tag found'. (The Length is 0x02.)\\
\hline
Slot 0-3 & Contains the message ``Header'', ``Reserved'', ``Length'' and\\
& ``Command''.\\
Slot 4 & Contains the Error Code.\\
& \indent - 0x4C: `L' Command in progress.\\
& \indent - 0x55: `U' Command in progress but RF field is off.\\
Slot 5 & Contains the Checksum.\\
\hline
\hline
&The RFID output in UART for Authenticating a Data block.\\
\hline
Slot 0-3 & Contains the message ``Header'', ``Reserved'', ``Length'' and\\
& ``Command''.\\
Slot 4 & Contains the Status/Error Code.\\
& \indent - 0x4C: `L' - Login Successfull.\\
& \indent - 0x4E: `N' - No Tag Present or Login Failed.\\
& \indent - 0x55: `U' - Login Failed.\\
& \indent - 0x45: `E' - Invalid Key format in E2PROM.\\
Slot 5 & Contains the Checksum.\\
\hline
\hline 
& The RFID output in UART for reading a block: On a success.\\
& (The length would be 0x12.) \\
\hline
Slot 0-3 & Contains the message ``Header'', ``Reserved'', ``Length'' and\\
& ``Command''.\\
Slot 4 & Contains the number of the block read. \\
Slot 5-20 & contains the data stored in the block.\\
Slot 21 & Contains the Checksum.\\
\hline
&On a Fail (The length is 0x02.)\\
\hline
Slot 0-3 & Contains the message ``Header'', ``Reserved'', ``Length'' and\\
& ``Command''.\\
Slot 4 & Contains the error code:\\
& \indent - 0x4E: `N' No tag present.\\
& \indent - 0x46: `F' Failed to read.\\
Slot 5 & Contains the Checksum.\\
\hline
\end{tabular}