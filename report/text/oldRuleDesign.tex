\chapter{Rules first design}
\label{appendixFirstRuleDesign}
Before the final design had been made we used the design in this chapter.

To see a quick overview of the different conditions and their name:
\begin{description}
	\item[Timeperiode] the action can be done in this time period. 
	\item[Timestamp] the action may only happen at a given point in time. 
	\item[Controller on] the action can be done if a specific controller is turn on. 
	\item[Controller off] the action can be done if a specific controller is turn off. 
	\item[True] The action can always be taken.
\end{description}

An overview of the actions and their name is listed below:

\begin{description}
	\item[Block user] it will block the profiles of all profiles connected to the rule.
	\item[Activate user] it will activate the profiles connected to it. 
	\item[Add points] it will add points to the profiles' points.
	\item[Delete points]  it will delete points to the profiles' points.  
	\item[Set maximum of point] it will set a maximum number of points that a profile can have. 
	\item[Unlimited time] it will give the profile unlimited time to be spend on any media. 
	\item[Access any controller] it will give the profile access to any media in the system. 
	\item[Cannot access any controller] the profile will not have access to any media. 
	\item[Access controller] it will give the profile access to a specific media. 
	\item[Cannot access controller] it will block the user from using a specific media. 
\end{description}

The rule's structure is presented in a grammar in listing	\ref{grammar} expressed in Extended Backus-Naur Form(EBNF) \citep{CoPL}.
In a nonterminal is en-captured in $<>$ and a terminal is just a word or en-captured in ``''. 
Also the grouping is used represented in $()$, the replica symbol is $*$, comments is $(**)$ and alternative is $|$. 

A rule consist of a name and one of five action and condition sets which determine which actions and conditions can be combined. A rule can have several actions and conditions but only from the same set, see line 1-7. The action set is presented from 9-23 where all has a specific name, some have a specific Controller represented as a number and some has points which is a number. The condition set likewise represented from 9-23 and the condition types are presented from 25-30. There are four types but they each have a specific name. One type has Timestamp, another a Controller, the third only the name and the last is a timeperiod. The timeperiod has with two timestamp and if it is repeatable it has a string representation of the weekdays and a representation of then it is repeatable. 
\begin{lstlisting}[frame=single, label=grammar, caption=Grammar of a rule in EBNF]
<Rule>:= <name> (
	 	  (<ActionsetSet1><ConditionSet1>)*
		| (<ActionsetSet2><ConditionSet2>)*
		| (<ActionsetSet3><ConditionSet3>)*
		| (<ActionsetSet4><ConditionSet4>)*
		| (<ActionsetSet5><ConditionSet5>)*
	)

<ActionsetSet1> := "Block user" | "Activate user" 
<ConditionSet1> := <ConditionTimestamp>

<ActionsetSet2> := ("Add points" | "Delete points") <Points>
<ConditionSet2> := <ConditionTimeperiod> | <ConditionTimestamp>

<ActionsetSet3> := "Set maximum of point" <Points>
<ConditionSet3> := <ConditionTrue>

<ActionsetSet4> := "Unlimited time" | ("Access any controller" | "Cannot access any controller") <Controller>
<ConditionSet4> := <ConditionTimeperiod> | <ConditionTimestamp> 

<ActionsetSet5> := ("Access controller" | "Cannot access controller")<Controller>
<ConditionSet5> := <ConditionTimestamp> |	<ConditionTimeperiod> 
					| 	<ConditionTrue>	|	<ConditionElse>	
				
<ConditionTimestamp> :=  "Timestamp" <Timestamp>
<ConditionTimeperiod> :=  "TimePeriod" <Timestamp> <Timestamp> <Repeatable>					
<Repeatable> :=   <Weekdays> <Repeat> | Nill

<ConditionTrue> := "True" 
<ConditionElse>:= ("Device on" | "Device off") <Controller>
				
<Name> := ALPHA*  (* Upper and lowercase ASCII letters (A-Z,a-z) *)
<Controller> := DIGIT* (* Decimal digits (0-9) *)
<Points> := DIGIT*
<Timestamp> := 4*DIGIT,"-",2*DIGIT,"-",2*DIGIT, " ",2*DIGIT,":"2*DIGIT,":"2*DIGIT  (*YYYY-MM-DD HH:mm:ss*)
<Weekdays> := ("monday"| "tuesday"| "wednesday"| "thursday"| "friday"| "saturday"| "sunday")*
<Repeat> := "weekly" | "biweekly" | "triweekly" | "first in month" | "last in month"
\end{lstlisting}
